\documentclass[aspectratio=169]{beamer}
\mode<presentation>
\usetheme{Hannover}
\useoutertheme{sidebar}
\usecolortheme{dolphin}

\usepackage{amsmath}
\usepackage{amssymb}
\usepackage{enumerate}

%some bold math symbosl
\newcommand{\Cov}{\mathrm{Cov}}
\newcommand{\Var}{\mathrm{Var}}
\newcommand{\Cor}{\mathrm{Cor}}
\newcommand{\brho}{\boldsymbol{\rho}}
\newcommand{\bSigma}{\boldsymbol{\Sigma}}
\newcommand{\btheta}{\boldsymbol{\theta}}
\newcommand{\bbeta}{\boldsymbol{\beta}}
\newcommand{\bmu}{\boldsymbol{\mu}}
\newcommand{\bW}{\mathbf{W}}
\newcommand{\one}{\mathbf{1}}
\newcommand{\bH}{\mathbf{H}}
\newcommand{\by}{\mathbf{y}}
\newcommand{\bolde}{\mathbf{e}}
\newcommand{\bx}{\mathbf{x}}

\newcommand{\cpp}[1]{\texttt{#1}}



\title{Mathematical Biostatistics Boot Camp: Lecture 4, Random Vectors }

\author{Brian Caffo}
\date{\today}
\institute[Department of Biostatistics]{
  Department of Biostatistics \\
  Johns Hopkins Bloomberg School of Public Health\\
  Johns Hopkins University
}


\begin{document}

\frame{\titlepage}


\section{Table of contents}
\frame{
  \frametitle{Table of contents}
  \tableofcontents
}

\section{Random vectors}
\begin{frame}\frametitle{Random vectors}
  \begin{itemize}
  \item Random vectors are simply random variables collected into a vector
    \begin{itemize}
    \item For example if $X$ and $Y$ are random variables $(X, Y)$ is a random vector
    \end{itemize}
  \item Joint density $f(x, y)$ satisfies $f > 0$ and $\int \int f(x, y) dx dy = 1$
  \item For discrete random variables $\sum \sum f(x, y) = 1$
  \item In this lecture we focus on {\bf independent} random variables where $f(x, y) = f(x)g(y)$
  \end{itemize}
\end{frame}

\section{Independence}
\subsection{Independent events}
\begin{frame}\frametitle{Independent events}
  \begin{itemize}
  \item Two events $A$ and $B$ are {\bf independent} if $$P(A \cap B) = P(A)P(B)$$
  \item Two random variables, $X$ and $Y$ are independent if for any two sets $A$ and $B$
    $$P([X \in A] \cap [Y \in B]) = P(X\in A)P(Y\in B)$$
  \item If $A$ is independent of $B$ then 
    \begin{itemize}
    \item $A^c$ is independent of $B$ 
    \item $A$ is independent of $B^c$
    \item $A^c$ is independent of $B^c$
    \end{itemize}
  \end{itemize}
\end{frame}


\begin{frame}\frametitle{Example}
  \begin{itemize}
  \item What is the probability of getting two consecutive heads?
  \item $A = \{\mbox{Head on flip 1}\}$ ~ $P(A) = .5$
  \item $B = \{\mbox{Head on flip 2}\}$ ~ $P(B) = .5$
  \item $A \cap B = \{\mbox{Head on flips 1 and 2}\}$
  \item $P(A \cap B) = P(A)P(B) = .5 \times .5 = .25$ 
  \end{itemize}
\end{frame}

\begin{frame} \frametitle{Example}
  \begin{itemize}
     \item Volume 309 of {\em Science} reports on a physician who was
       on trial for expert testimony in a criminal trial
     \item Based on an estimated prevalence of sudden infant death
       syndrome of $1$ out of $8,543$, Dr Meadow testified that 
       that the probability of a mother having two children with SIDS
       was $\left(\frac{1}{8,543}\right)^2$
     \item The mother on trial was convicted of murder
     \item What was Dr Meadow's mistake(s)?
  \end{itemize}
\end{frame}

\begin{frame}\frametitle{Example: continued}
  \begin{itemize}
    \item For the purposes of this class, the principal mistake was to
      {\em assume} that the probabilities of having SIDs within a family are
      independent
    \item That is, $P(A_1 \cap A_2)$ is not necessarily equal to $P(A_1)P(A_2)$
    \item Biological processes that have a believed genetic or familiar
      environmental component, of course, tend to be dependent within families
    \item In addition, the estimated prevalence was obtained from an
      {\em unpublished} report on single cases; hence having no information about
      recurrence of SIDs within families
  \end{itemize}
  
\end{frame}
\subsection{Independent random variables}
\begin{frame}\frametitle{Useful fact}
  \begin{itemize}
  \item   We will use the following fact extensively in this class:
  \end{itemize}
  \begin{quote}
    If a collection of random variables $X_1, X_2, \ldots, X_n$ are
    independent, then their joint distribution is the product of their
    individual densities or mass functions \ \\ \ \\
    
    That is, if $f_i$ is the
    density for random variable $X_i$ we have that
    $$
    f(x_1,\ldots, x_n) = \prod_{i=1}^n f_i(x_i)
    $$
      \end{quote}
\end{frame}


\subsection{IID random variables}
\begin{frame}\frametitle{IID random variables}
  \begin{itemize}
  \item In the instance where $f_1 = f_2 = \ldots = f_n$ we say that the $X_i$ are {\bf iid}
    for {\em independent} and {\em identically distributed}
  \item iid random variables are the default model for random samples
  \item Many of the important theories of statistics are founded on assuming that
    variables are iid
  \end{itemize}
\end{frame}


\begin{frame}
  \frametitle{Example}
  \begin{itemize}
  \item Suppose that we flip a biased coin with success probability $p$ $n$
  times, what is the join density of the collection of
  outcomes?
  \item These random variables are iid with densities $p^{x_i} (1 - p)^{1-x_i}$ 
  \item Therefore
  $$
  f(x_1,\ldots,x_n) = \prod_{i=1}^n p^{x_i} (1 - p)^{1-x_i} = p^{\sum x_i} (1 - p)^{n - \sum x_i}
  $$
\end{itemize}
\end{frame}

\section{Correlation}
\begin{frame}\frametitle{Correlation}
  \begin{itemize}
  \item The {\bf covariance} between two random variables $X$ and $Y$ is defined
    as 
    $$
    \Cov(X, Y) = E[(X - \mu_x)(Y - \mu_y)] = E[X Y] - E[X]E[Y]
    $$
  \item The following are useful facts about covariance
    \begin{enumerate}
    \item $\Cov(X, Y) = \Cov(Y, X)$
    \item $\Cov(X, Y)$ can be negative or positive
    \item $|\Cov(X, Y)| \leq \sqrt{\Var(X) \Var(y)}$
    \end{enumerate}
  \end{itemize}
\end{frame}

\begin{frame}\frametitle{Correlation}
  \begin{itemize}
  \item The {\bf correlation} between $X$ and $Y$ is 
    $$
    \Cor(X, Y) = \Cov(X, Y) / \sqrt{\Var(X) \Var(y)}
    $$
    \begin{enumerate}
    \item $-1 \leq \Cor(X, Y) \leq 1$
    \item $\Cor(X, Y) = \pm 1$ if and only if $X = a + bY$ for some constants $a$ and $b$
    \item $\Cor(X, Y)$ is unitless
    \item $X$ and $Y$ are {\bf uncorrelated} if $\Cor(X, Y) = 0$ 
    \item $X$ and $Y$ are more positively correlated, the closer $\Cor(X,Y)$ is to $1$
    \item $X$ and $Y$ are more negatively correlated, the closer $\Cor(X,Y)$ is to $-1$
    \end{enumerate}
  \end{itemize}
\end{frame}

\section{Variance and correlation properties}
\begin{frame}\frametitle{Some useful results}
  \begin{itemize}
  \item Let $\{X_i\}_{i=1}^n$ be a collection of random variables
  \begin{itemize}
  \item When the $\{X_i\}$ are uncorrelated 
    $$\Var\left(\sum_{i=1}^n a_i X_i + b\right) = \sum_{i=1}^n a_i^2
    \Var(X_i)$$  
  \item Otherwise
    \begin{eqnarray*}
&   &  \Var\left(\sum_{i=1}^n a_i X_i + b\right)\\
& = &\sum_{i=1}^n a_i^2 \Var(X_i) + 2 \sum_{i=1}^{n-1} \sum_{j=i}^n a_i a_j \Cov(X_i, X_j)
    \end{eqnarray*}
  \item If the $X_i$ are iid with variance $\sigma^2$ then $\Var(\bar X) = \sigma^2/n$ and
    $E[S^2] = \sigma^2$
  \end{itemize}
\end{itemize}
\end{frame}

\begin{frame}\frametitle{Example proof}
  Prove that $\Var(X+Y) = \Var(X) + \Var(Y) + 2 \Cov(X, Y)$
\hspace{-1in}  \begin{eqnarray*}
    &   & \Var(X + Y) \\ \\
    & = & E[(X + Y)(X + Y)] - E[X + Y]^2 \\ \\
    & = & E[X^2 + 2XY + Y^2] - (\mu_x + \mu_y)^2 \\ \\
    & = & E[X^2 + 2XY + Y^2] - \mu_x^2 - 2\mu_x\mu_y -\mu_y^2 \\ \\
    & = & (E[X^2]-\mu_x^2) + (E[Y^2] - \mu_y^2) + 2(E[XY] - \mu_x\mu_y)\\ \\
    & = & \Var(X) + \Var(Y) + 2 \Cov(X, Y)
  \end{eqnarray*}
\end{frame}

\begin{frame}
  \frametitle{Result}
  \begin{itemize}
  \item A commonly used subcase from these properties is that {\em if a collection of 
    random variables $\{X_i\}$ are uncorrelated}, then the variance of the sum is the
    sum of the variances
    $$
    \Var\left(\sum_{i=1}^n X_i \right) = \sum_{i=1}^n \Var(X_i)
    $$
  \item Therefore, it is sums of variances that tend to be useful, not
    sums of standard deviations; that is, the standard deviation of the
    sum of bunch of independent random variables is the square root of the
    sum of the variances, not the sum of the standard deviations
  \end{itemize}
\end{frame}

\section{Variances properties of sample means}
\begin{frame}\frametitle{The sample mean}
  Suppose $X_i$ are iid with variance $\sigma^2$
  \begin{eqnarray*}
    \Var(\bar X) & = & \Var \left( \frac{1}{n}\sum_{i=1}^n X_i \right)\\ \\
    & = & \frac{1}{n^2} \Var\left(\sum_{i=1}^n X_i \right)\\ \\
    & = & \frac{1}{n^2} \sum_{i=1}^n \Var(X_i) \\ \\
    & = & \frac{1}{n^2} \times n\sigma^2 \\ \\
    & = & \frac{\sigma^2}{n}
  \end{eqnarray*}
\end{frame}


\begin{frame}\frametitle{Some comments}
  \begin{itemize}
  \item When $X_i$ are independent with a common variance $\Var(\bar X) = \frac{\sigma^2}{n}$
  \item $\sigma/\sqrt{n}$ is called {\bf the standard error} of the sample mean
  \item The standard error of the sample mean is the standard deviation of the distribution of the sample mean
  \item $\sigma$ is the standard deviation of the distribution of a single observation
  \item Easy way to remember, the sample mean has to be less variable than a single observation, therefore its
    standard deviation is divided by a $\sqrt{n}$
  \end{itemize}
\end{frame}

\section{The sample variance}
\begin{frame}\frametitle{The sample variance}
  \begin{itemize}
  \item The {\bf sample variance} is defined as 
    $$
    S^2 =   \frac{\sum_{i=1}^n (X_i - \bar X)^2}{n-1} 
    $$
  \item The sample variance is an estimator of $\sigma^2$
  \item The numerator has a version that's quicker for calculation
    $$
    \sum_{i=1}^n (X_i - \bar X)^2 = \sum_{i=1}^n X_i^2 - n \bar X^2
    $$
  \item The sample variance is (nearly) the mean of the squared deviations from the mean
  \end{itemize}
\end{frame}

\begin{frame}\frametitle{The sample variance is unbiased}
  \begin{eqnarray*}
    E\left[\sum_{i=1}^n (X_i - \bar X)^2\right] & = & \sum_{i=1}^n E\left[X_i^2\right] - n E\left[\bar X^2\right] \\ \\
    & = & \sum_{i=1}^n \left\{\Var(X_i) + \mu^2\right\} - n \left\{\Var(\bar X) + \mu^2\right\} \\ \\
    & = & \sum_{i=1}^n \left\{\sigma^2 + \mu^2\right\} - n \left\{\sigma^2 / n + \mu^2\right\} \\ \\
    & = & n \sigma^2 + n \mu ^ 2 - \sigma^2 - n \mu^2 \\ \\
    & = & (n - 1) \sigma^2
  \end{eqnarray*}
\end{frame}

\section{Some discussion}
\begin{frame}\frametitle{Hoping to avoid some confusion}
  \begin{itemize}
  \item Suppose $X_i$ are iid with mean $\mu$ and variance $\sigma^2$
  \item $S^2$ estimates $\sigma^2$
  \item The calculation of $S^2$ involves dividing by $n-1$
  \item $S / \sqrt{n}$ estimates $\sigma / \sqrt{n}$ the standard error of the mean
  \item $S / \sqrt{n}$ is called the sample standard error (of the mean)
  \end{itemize}
\end{frame} 

\begin{frame}\frametitle{Example} 
  \begin{itemize}
  \item In a study of 495 organo-lead workers, the
    following summaries were obtained for TBV in $cm^3$
  \item mean = $1151.281$
  \item sum of squared observations = $662361978$
  \item sample sd = $\sqrt{(662361978 - 495 \times 1151.281^2)/494} = 112.6215$
  \item estimated se of the mean = $112.6215 / \sqrt{495} = 5.062$
  \end{itemize}
\end{frame}

\end{document}

